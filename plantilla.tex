\documentclass[11pt,letterpaper]{article} %Formato general del documento
\usepackage[spanish]{babel} % Idioma
\usepackage[utf8]{inputenc} % Codificación
\usepackage[titletoc,title]{appendix} % Añadir apendices al indice
\usepackage{graphicx} % Agregar figuras(imágenes)
\usepackage{float} % Solución para imágenes flotantes
\usepackage{xcolor} % Colores en codigo fuente
\usepackage{listings} % Importar códigos fuente
\usepackage{hyperref} % Crear enlaces de indice + glosario + acrónimos
\usepackage{xparse}
\usepackage[acronym,xindy,toc]{glossaries} % Glosario + Acronimos
\makeglossaries
\usepackage[xindy]{imakeidx}
\makeindex

\newcommand{\set}[2]{\newcommand{#1}{#2}}

\newcommand{\init}{
  \addto{\captionsspanish}{\renewcommand*{\contentsname}{\tocName}} %Renombrar TOC
  \setcounter{secnumdepth}{-2} % Quitar numeración de secciones
  \loadglsentries[main]{glosario.tex} % Agregar archivo de glosario
  \setglossarystyle{altlisthypergroup} % Estilo de glosario/acrónimos
  \renewcommand{\lstlistingname}{\codeCaption{}}

  \title{
    \textbf{\nombreDeLaUniversidad}\\
    \nombreDelDepartamento\\
    \begin{figure}[ht]
      \centering
      \includegraphics[width=\largoDelLogo]{\logoDeLaUniversidad}
    \end{figure}
    \textbf{\tipoDeActividad{}}\\
    \tituloDelDocumento{}\\
    \textit{\materia{}}\\
  }
  \author{\nombreDelAutor{}\\\emailDelAutor{}}

  \lstset{literate=
    {á}{{\'a}}1 {é}{{\'e}}1 {í}{{\'i}}1 {ó}{{\'o}}1 {ú}{{\'u}}1
    {Á}{{\'A}}1 {É}{{\'E}}1 {Í}{{\'I}}1 {Ó}{{\'O}}1 {Ú}{{\'U}}1
    {à}{{\`a}}1 {è}{{\`e}}1 {ì}{{\`i}}1 {ò}{{\`o}}1 {ù}{{\`u}}1
    {À}{{\`A}}1 {È}{{\'E}}1 {Ì}{{\`I}}1 {Ò}{{\`O}}1 {Ù}{{\`U}}1
    {ä}{{\"a}}1 {ë}{{\"e}}1 {ï}{{\"i}}1 {ö}{{\"o}}1 {ü}{{\"u}}1
    {Ä}{{\"A}}1 {Ë}{{\"E}}1 {Ï}{{\"I}}1 {Ö}{{\"O}}1 {Ü}{{\"U}}1
    {â}{{\^a}}1 {ê}{{\^e}}1 {î}{{\^i}}1 {ô}{{\^o}}1 {û}{{\^u}}1
    {Â}{{\^A}}1 {Ê}{{\^E}}1 {Î}{{\^I}}1 {Ô}{{\^O}}1 {Û}{{\^U}}1
    {œ}{{\oe}}1 {Œ}{{\OE}}1 {æ}{{\ae}}1 {Æ}{{\AE}}1 {ß}{{\ss}}1
    {ű}{{\H{u}}}1 {Ű}{{\H{U}}}1 {ő}{{\H{o}}}1 {Ő}{{\H{O}}}1
    {ç}{{\c c}}1 {Ç}{{\c C}}1 {ø}{{\o}}1 {å}{{\r a}}1 {Å}{{\r A}}1
    {€}{{\EUR}}1 {£}{{\pounds}}1
  }
  \lstdefinestyle{customc}{
    belowcaptionskip=1\baselineskip,
    breaklines=true,
    frame=L,
    xleftmargin=\parindent,
    language=C,
    showstringspaces=false,
    basicstyle=\footnotesize\ttfamily,
    keywordstyle=\bfseries\color{green!40!black},
    commentstyle=\itshape\color{purple!40!black},
    identifierstyle=\color{blue},
    stringstyle=\color{orange}
  }
}
\newcommand{\code}[1]{
  \lstinputlisting[caption=#1, escapechar=, style=customc]{#1}
}

\newcommand{\documento}[1]{
  \begin{document}
  \maketitle
  \newpage

  \tableofcontents
  \newpage

  \printindex
  \printglossary[title=\glossaryName{},toctitle=\glossaryName{}]
  \newpage
  \printglossary[type=\acronymtype,title=\acronName{},toctitle=\acronName{}]
  \newpage

  #1

  \end{document}
}

\newcommand{\resumen}[1]{
  \begin{abstract}
    #1
  \end{abstract}
}

\newcommand{\introduccion}[1]{\section{Introducción}#1}
\newcommand{\objetivo}[1]{\section{Objetivo}#1}
\newcommand{\desarrollo}[1]{\section{Desarrollo}#1}
\newcommand{\resultados}[1]{\section{Resultados}#1}
\newcommand{\apendice}[1]{
  \begin{appendices}
    #1
    \renewcommand{\refname}{\section{\bibName{}}}
    \bibliographystyle{plain} \bibliography{bibliografia.bib}
  \end{appendices}
}

\set{\bibName}{Bibliografia}
\set{\acronName}{Acrónimos}
\set{\glossaryName}{Glosario}
\set{\tocName}{Tabla de contenido}
\set{\codeCaption}{Código}

\DeclareDocumentCommand{\newdualentry}{O{} O{} m m m m}{
    \newglossaryentry{gls-#3}{name={#5},text={#5\glsadd{#3}},
      description={#6},#1
    }
    \newacronym[see={[\glossaryName:]{gls-#3}},#2]{#3}{#4}{#5\glsadd{gls-#3}}
}

% Local IspellDict: espanol


\set{\logoDeLaUniversidad}{figures/UDG_byn.png}
\set{\largoDelLogo}{7cm}
\set{\nombreDeLaUniversidad}{Universidad de Guadalajara}
\set{\nombreDelDepartamento}{Departamento de electrónica}
\set{\tipoDeActividad}{Reporte}
\set{\tituloDelDocumento}{Plantilla \LaTeX}
\set{\materia}{Automatizacion de actividades rutinarias}
\set{\nombreDelAutor}{Eduardo Hao}
\set{\emailDelAutor}{lalohao@elmejorproveedor.com}

\init{}
\documento{
  \resumen{

    Se trató de simplificar algunas funciones de latex de manera que se
    maximizará la utilizad y a la vez se redujera la complejidad de la
    creación de un archivo \LaTeX .

  }

  \introduccion{

    Siempre es difícil aprender una nueva tecnología por lo que me
    dispuse a realizar este pequeño documento que ayudará (a mi y espero
    que a ustedes también) a crear fácilmente artículos.

  }

  \objetivo{

    Demostrar las capacidades de \LaTeX de una manera simple y eficiente.

  }

  \desarrollo{

    Ver {glosario.tex, bibliografia.bib} para modificar el glosario y
    bibliografía respectivamente.

    Para darle otro formato al documento revisar el archivo estilo.tex

  }

  \resultados{

    El uso de algunas referencias: por ejemplo esta cita
    \cite{yonson98:_el}, o esta otra donde viene el esquemático
    \cite{pic16_datasheet}



    Glosarios:


    \gls{pi} es un \gls{numero real}; $4$, $5$, y $6$ también son
    \glspl{numero real}, además \gls{lol} es un acrónimo.

    Quizás después este documento sea aún mas sencillo \gls{lol}.



    Además claro esta lo básico como índice, ecuaciones y enlaces a
    partes internas del documento

  }

  \apendice{
    \section{Código fuente}
    \code{hola.c}
  }
}
